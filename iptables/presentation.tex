\documentclass[14pt]{beamer}
\usepackage{hyperref}
\usepackage{ulem}
\usepackage[T1]{fontenc}
\usetheme{Madrid}
\title[Reduce iptables config complexity]{Reducing iptables configuration complexity using chains}
\author{Dieter Adriaenssens}
\institute[UGent]{Ghent University, Belgium}
\date[LinuxTag 8May2014]{LinuxTag - Berlin\\
May 8th, 2014}
\subject{iptables}
\begin{document}
  \begin{frame}
  \titlepage
  \end{frame}
  \begin{frame}
    \frametitle{Overview}
    \begin{itemize}
      \item introduction to iptables
      \item optimizing configuration with an example setup
    \end{itemize}
  \end{frame}
  \begin{frame}
    \frametitle{What is iptables?}
    %(question to audience : who knows/uses iptables?)
    Iptables is a tool for creating the rulesets for netfilter, a packet filtering framework which was introduced in the linux 2.4 kernel
  \end{frame}
  \begin{frame}
    \frametitle{Rules}
    When an IP packet comes in, it is checked against a set of rules.
    \begin{example}
      \small{iptables -A INPUT -m tcp -p tcp --dport ssh -j ACCEPT\\
      iptables -A INPUT -m tcp -p tcp --dport http -j ACCEPT\\
      iptables -A INPUT -m tcp -p tcp --dport https -j ACCEPT\\
      iptables -A INPUT -j DROP}
    \end{example}
  \end{frame}
  \begin{frame}
    \frametitle{Chains}
    Predefined chains :
    \begin{itemize}
      \item INPUT
      \item OUTPUT
      \item FORWARD
      \item PREROUTING
      \item POSTROUTING
    \end{itemize}
    \begin{example}
      \small{-A \textbf{INPUT} -m tcp -p tcp --dport ssh -j ACCEPT}
    \end{example}
    You can define your own chains
  \end{frame}
  \begin{frame}
    \frametitle{Filters}
    Filter on packet parameters :
    \begin{itemize}
      \item protocol (tcp, udp, icmp, \ldots)
      \item destination/source port
      \item destination/source IP address
      \item in/outgoing interface (eth0, \ldots)
      \item \ldots
    \end{itemize}
    \begin{example}
      \small{-A INPUT \textbf{-m tcp -p tcp --dport ssh -s 10.0.0.0/8} -j ACCEPT}
    \end{example}
  \end{frame}
  \begin{frame}
    \frametitle{Targets}
    What to do if a packet matches a rule :
    \begin{itemize}
      \item ACCEPT
      \item DROP
      \item QUEUE $\rightarrow$ userspace
      \item RETURN $\rightarrow$ leave current chain
      \item LOG
      \item jump to a custom chain
    \end{itemize}
    \begin{example}
      \small{-A INPUT -m tcp -p tcp --dport ssh \textbf{-j ACCEPT}}
    \end{example}
  \end{frame}
  \begin{frame}
    \frametitle{Putting it all together}
    All rules are checked one for one, and if one matches the target is executed :
    \begin{example}
      \small{-A INPUT -m tcp -p tcp --dport ssh -j ACCEPT\\
      -A INPUT -m tcp -p tcp --dport 80 -j ACCEPT\\
      -A INPUT -m tcp -p tcp --dport https -j ACCEPT\\
      -A INPUT -j DROP}
    \end{example}
    The last rule matches everything, so if a packet didn't match a previous rule, it will be rejected.\\
    \pause
    \textbf{Remark:} the order of the rules is important
 \end{frame}
  \begin{frame}
    \frametitle{Default target DROP}
    But you can define default target for the predefined chains:
    \begin{example}
      \small{{\color{blue}-P INPUT DROP}\\
      -A INPUT -m tcp -p tcp --dport ssh -j ACCEPT\\
      -A INPUT -m tcp -p tcp --dport 80 -j ACCEPT\\
      -A INPUT -m tcp -p tcp --dport https -j ACCEPT\\
      \sout{{\color{red}-A INPUT -j DROP}}}
    \end{example}
    If none of the rules match, the default target is executed.\\
    In this example, a packet is dropped.
  \end{frame}
  \begin{frame}
    \frametitle{Default target ACCEPT}
    \textbf{Exercise:}\\
    What happens if the default target is ACCEPT?
    \begin{example}
      \small{-P INPUT ACCEPT\\
      -A INPUT -m tcp -p tcp --dport ssh -j ACCEPT\\
      -A INPUT -m tcp -p tcp --dport 80 -j ACCEPT\\
      -A INPUT -m tcp -p tcp --dport https -j ACCEPT}
    \end{example}
    \pause
    All packets are accepted, even if no rule matches them.\\
    \pause
    \textbf{Always define a default target or use an all-matching rule at the end!}
  \end{frame}
  \begin{frame}
    \frametitle{Firewall for your server}
    Filter both incoming and outgoing traffic :
    \begin{example}
      \small{-P INPUT DROP\\
      -P OUTPUT DROP\\
      -A INPUT -m tcp -p tcp --dport ssh -j ACCEPT\\
      -A INPUT -m tcp -p tcp --dport 80 -j ACCEPT\\
      -A INPUT -m tcp -p tcp --dport https -j ACCEPT}
    \end{example}
    \pause
    \textbf{Problem:} incoming packets are accepted, but replies are dropped!
  \end{frame}
  \begin{frame}
    \frametitle{Firewall for your server}
    Also add rules for outgoing traffic :
    \begin{example}
      \small{-P INPUT DROP\\
      -P OUTPUT DROP\\
      -A INPUT -m tcp -p tcp --dport ssh -j ACCEPT\\
      -A INPUT -m tcp -p tcp --dport 80 -j ACCEPT\\
      -A INPUT -m tcp -p tcp --dport https -j ACCEPT\\
      -A OUTPUT -m tcp -p tcp --sport ssh -j ACCEPT\\
      -A OUTPUT -m tcp -p tcp --sport 80 -j ACCEPT\\
      -A OUTPUT -m tcp -p tcp --sport https -j ACCEPT}
    \end{example}
    \pause
    This can get complicated if more rules are added.
  \end{frame}
  \begin{frame}
    \frametitle{Established state}
    \textbf{Solution:} check for established state
    \begin{example}
      \small{-P INPUT DROP\\
      -P OUTPUT DROP\\
      -A INPUT -m tcp -p tcp --dport ssh -j ACCEPT\\
      -A INPUT -m tcp -p tcp --dport 80 -j ACCEPT\\
      -A INPUT -m tcp -p tcp --dport https -j ACCEPT\\
      -A OUTPUT -m state --state ESTABLISHED,RELATED -j ACCEPT}
    \end{example}
    This makes the rules more manageable, especially when output rules are also defined.
  \end{frame}
  \begin{frame}
    \frametitle{Intermezzo : firewall on your PC}
    Protect your own PC, block all incoming requests :
    \begin{example}
      \small{-P INPUT DROP\\
      -P OUTPUT ACCEPT}\\
      \pause
      \small{-A INPUT -i lo -j ACCEPT \# allow local traffic\\
      -A INPUT -p icmp -j ACCEPT \# allow ping\\
      -A INPUT -m state --state ESTABLISHED,RELATED -j ACCEPT}
    \end{example}
    Add ESTABLISHED, to allow replies of outgoing connections you intiated!
  \end{frame}
  \begin{frame}
    \frametitle{Tutorial example : setup}
    \begin{itemize}
      \item Tcp port 80 (http) and 443 (https) is available for all.
      \item SSH (port 22) and a webbased admin tool (port 10000), is limited to admin PCs.
      \item A SMB service is limited to admin and webmaster PCs.
      %\item Some outgoing connections are allowed (dns, dhcp, smtp, ntp).
      %\item Some external websites are reachable.
    \end{itemize}
  \end{frame}
  \begin{frame}
    \frametitle{Tutorial example : webserver}
    Webservice, tcp port 80 (http) and 443 (https) is available for all :
    \begin{example}
      \small{-A INPUT -m tcp -p tcp --dport 80 -j ACCEPT\\
      -A INPUT -m tcp -p tcp --dport https -j ACCEPT}
    \end{example}
  \end{frame}
  \begin{frame}
    \frametitle{Tutorial example : sysadmin PCs}
    Sysadmins are allowed to access SSH (tcp port 22) and a webbased admin tool (tcp port 10000).\\
    The IP addresses of their PCs :
    \begin{itemize}
	\item 10.100.2.3
	\item 10.100.2.4
	\item 10.100.2.7
    \end{itemize}
  \end{frame}
  \begin{frame}
    \frametitle{Tutorial example : sysadmin PCs}
    \begin{example}
      \small{-A INPUT -p tcp -m tcp -s 10.100.2.3 --dport 22 -j ACCEPT\\
      -A INPUT -p tcp -m tcp -s 10.100.2.4 --dport 22 -j ACCEPT\\
      -A INPUT -p tcp -m tcp -s 10.100.2.7 --dport 22 -j ACCEPT\\
      -A INPUT -p tcp -m tcp -s 10.100.2.3 --dport 10000 -j ACCEPT\\
      -A INPUT -p tcp -m tcp -s 10.100.2.4 --dport 10000 -j ACCEPT\\
      -A INPUT -p tcp -m tcp -s 10.100.2.7 --dport 10000 -j ACCEPT}
    \end{example}
    \pause
    The same IP addresses are repeated. If there is a change in the IP addresses list, several rules have to be updated.\\
    \pause
    \textbf{This can be done more efficient!}
  \end{frame}
  \begin{frame}
    \frametitle{Tutorial example : sysadmin PCs}
    Wouldn't it be convenient to :
    \begin{itemize}
      \item make a list of IP addresses?
      \item check this list when a packet matches a rule (fe. TCP port 22)?
      \item reuse this list when another rule matches a packet?
    \end{itemize}
    \pause
    \textbf{This is possible with chains!}
  \end{frame}
  \begin{frame}
    \frametitle{Introducing chains}
    Create the chain and add the rules
    \begin{example}
      \small{-N admin\_IP\\
      \pause
      -A admin\_IP -s 10.100.2.3 -j ACCEPT\\
      -A admin\_IP -s 10.100.2.4 -j ACCEPT\\
      -A admin\_IP -s 10.100.2.7 -j ACCEPT\\
      -A admin\_IP -j DROP}
    \end{example}
    \pause
    \begin{itemize}[<+->]
      \item Syntax of adding rules to a custom chain is similar to adding to default targets.
      \item Custom chains don't have a default target, so set a target for all packets that are not matched.
    \end{itemize}
  \end{frame}
  \begin{frame}
    \frametitle{Tutorial example : sysadmin PCs}
    Add the rules for TCP port 22 and 10000 using the custom chain :
    \begin{example}
      \small{-A INPUT -p tcp -m tcp --dport 22 -j admin\_IP\\
      -A INPUT -p tcp -m tcp --dport 10000 -j admin\_IP}
    \end{example}
  \end{frame}
  \begin{frame}
    \frametitle{Benefits}
    \begin{itemize}
      \item the list of IP addresses in the custom chain is reused for both ports, so they have to be defined only once.
      \item adding/changing/removing an IP address is much easier
      \item a better overview of the firewall rules.
    \end{itemize}
  \end{frame}
  \begin{frame}
    \frametitle{Tutorial example : webmaster PCs}
    Let's do the same for webmasters AND admins having access to SMB.\\
    \pause
    Create a chain with webmaster IP addresses :
    \begin{example}
      \small{-N webmaster\_IP\\
      -A webmaster\_IP -s 10.100.2.11 -j ACCEPT\\
      -A webmaster\_IP -s 10.100.2.17 -j ACCEPT\\
      -A webmaster\_IP -s 10.100.2.34 -j ACCEPT\\
      -A webmaster\_IP -j DROP}
    \end{example}
  \end{frame}
  \begin{frame}
    \frametitle{Tutorial example : access to SMB}
    Add the rules for SMB (tcp port 139 and 445) using the custom chains :
    \begin{example}
      \small{-A INPUT -p tcp -m tcp --dport 445 -j admin\_IP\\
      -A INPUT -p tcp -m tcp --dport 445 -j webmaster\_IP\\
      -A INPUT -p tcp -m tcp --dport 139 -j admin\_IP\\
      -A INPUT -p tcp -m tcp --dport 139 -j webmaster\_IP}
    \end{example}
  \pause
  \textbf{This can be optimized!}
  \end{frame}
  \begin{frame}
    \frametitle{Chaining chains}
    Include the sysadmin chain in the webmaster chain:
    \begin{example}
      \small{-N webmaster\_IP\\
      -A webmaster\_IP -s 10.100.2.11 -j ACCEPT\\
      -A webmaster\_IP -s 10.100.2.17 -j ACCEPT\\
      -A webmaster\_IP -s 10.100.2.34 -j ACCEPT\\
      \sout{-A webmaster\_IP -j DROP}\\
      -A webmaster\_IP -j admin\_IP}
    \end{example}
    Jump to the sysadmin chain at the end of the webmaster chain.\\
    \textbf{Warning :} check conditions of ending chains
  \end{frame}
  \begin{frame}
    \frametitle{Tutorial example : access to SMB}
    The rules for SMB (tcp port 139 and 445) can be reduced to :
    \begin{example}
      \small{\sout{-A INPUT -p tcp -m tcp --dport 445 -j admin\_IP}\\
      -A INPUT -p tcp -m tcp --dport 445 -j webmaster\_IP\\
      \sout{-A INPUT -p tcp -m tcp --dport 139 -j admin\_IP}\\
      -A INPUT -p tcp -m tcp --dport 139 -j webmaster\_IP}
    \end{example}
  The admin chain is checked as well because the webmaster chain includes it.
  \end{frame}

  \begin{frame}
    \frametitle{Conclusion}
    \begin{itemize}
      \item use ESTABLISHED state to reduce number of rules
      \item using chains makes your rules easier readable and maintainable
      \item chains can be reused for several rules
      \item chains can be chained together
      \item faster, because it only jumps to a chain when a rule matches
    \end{itemize}
  \end{frame}
  \begin{frame}
   \frametitle{Questions}
    Thanks for your attention!\\
    Questions?\\
    Contact
    \begin{itemize}
      \item Blog : \href{http://ruleant.blogspot.com/}{http://ruleant.blogspot.com}
      \item Twitter : \href{https://twitter.com/dcadriaenssens}{@dcadriaenssens}
    \end{itemize}
  \end{frame}
\end{document}
