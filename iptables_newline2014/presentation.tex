\documentclass[14pt]{beamer}
\usepackage{hyperref}
\usepackage[T1]{fontenc}
\usetheme{Madrid}
\title[Reduce iptables config complexity]{Reducing iptables configuration complexity using chains}
\author{Dieter Adriaenssens}
\date[29Mar2014]{Newline 0x04\\
Whitespace, Ghent\\
March 29th, 2014}
\subject{iptables}
\begin{document}
  \begin{frame}
  \titlepage
  \end{frame}
  \begin{frame}
    \frametitle{What is iptables?}
    %(question to audience : who knows/uses iptables?)
    iptables is a tool for creating the rulesets for netfilter, a packet filtering framework which was introduced in the linux 2.4 kernel
  \end{frame}
  \begin{frame}
    \frametitle{Rules}
    When an IP packet comes in, it is checked against a set of rules.
    \begin{example}
      \small{iptables -A INPUT -p tcp --dport ssh -j ACCEPT\\
      iptables -A INPUT -p tcp --dport http -j ACCEPT\\
      iptables -A INPUT -p tcp --dport https -j ACCEPT\\
      iptables -A INPUT -j REJECT}
    \end{example}
  \end{frame}
  \begin{frame}
    \frametitle{Chains}
    Predefined chains :
    \begin{itemize}
      \item INPUT
      \item OUTPUT
      \item FORWARD
      \item PREROUTING
      \item POSTROUTING
    \end{itemize}
    You can define your own chains
  \end{frame}
  \begin{frame}
    \frametitle{Filters}
    Filter on packet parameters :
    \begin{itemize}
      \item protocol (tcp, udp, icmp, \ldots)
      \item destination/source port
      \item destination/source IP address
      \item in/outgoing interface (eth0, \ldots)
      \item \ldots
    \end{itemize}
    \begin{example}
      \small{iptables -A INPUT -p tcp --dport ssh -s 10.0.0.0/8 -j ACCEPT}
    \end{example}
  \end{frame}
  \begin{frame}
    \frametitle{Targets}
    What to do if a packet matches a rule :
    \begin{itemize}
      \item ACCEPT
      \item DROP
      \item QUEUE $\rightarrow$ userspace
      \item RETURN $\rightarrow$ leave current chain
      \item LOG
      \item jump to a custom chain
    \end{itemize}
  \end{frame}
  \begin{frame}
    \frametitle{Putting it all together}
    \begin{example}
      \small{iptables -A INPUT -p tcp --dport ssh -j ACCEPT\\
      iptables -A INPUT -p tcp --dport 80 -j ACCEPT\\
      iptables -A INPUT -p tcp --dport https -j ACCEPT\\
      iptables -A INPUT -j REJECT}
    \end{example}
    But you can define default behaviour for a predefined chain:
    \begin{example}
      \small{iptables -P INPUT DROP}
    \end{example}
  \end{frame}
  \begin{frame}
   \frametitle{Questions}
    Thanks for your attention!\\
    Questions?\\
    Contact
    \begin{itemize}
      \item Blog : \href{http://ruleant.blogspot.com/}{http://ruleant.blogspot.com}
      \item Twitter : \href{https://twitter.com/dcadriaenssens}{@dcadriaenssens}
    \end{itemize}
  \end{frame}
\end{document}
