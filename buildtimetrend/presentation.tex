\documentclass[14pt]{beamer}
\usepackage{hyperref}
\usepackage{ulem}
\usepackage[T1]{fontenc}
\usetheme{Madrid}

% macro for double dash
\newcommand{\dd}{{\texttt{-{}-}}}

%% Enable license helpers
\graphicspath{{../cc_beamer/}}
\input{../cc_beamer/cc_beamer}

\logo{
  \includegraphics[scale=.1]{pictures/logo_buildtimetrend.png}
}
\title[Buildtime Trend : What, why and how]{Buildtime Trend : What's trending in your build process}
\author{Dieter Adriaenssens}
\institute[Buildtime Trend]{Buildtime Trend founder, developer - @dcadriaenssens}
\date[NewLine 5Apr2015]{NewLine - Ghent\\
April 5th, 2015}
\subject{Buildtime Trend}
\begin{document}
  \begin{frame}
    \titlepage
    \vfill
    \begin{center}
      \CcGroupByNcSa{0.83}{0.95ex}\\[2.5ex]
        {\tiny\CcNote{\CcLongnameByNcSa}}
        \vspace*{-2.5ex}
    \end{center}
  \end{frame}
  \begin{frame}
    \frametitle{Overview}
    \begin{itemize}
      \item What is Buildtime Trend
      \item Why and how
      \item Demo
      \item Lessons learned
    \end{itemize}
  \end{frame}
  \begin{frame}
    \frametitle{What is Buildtime Trend?}
    %(question to audience : who knows/uses iptables?)
  \end{frame}
  \begin{frame}
    \frametitle{It started with an itch}
    \begin{itemize}
      \item Some builds took longer than others
      \item no timing information was present in the logs
      \item which stage took longer?
    \end{itemize}
    \pause
    \textbf{Solution:} create a script to generate timestamps
  \end{frame}
  \begin{frame}
    \frametitle{Next step : calculate duration and generate chart}
    Requirements :
    \begin{itemize}
      \item script to parse the generated timestamps
      \item calculate duration of the stages
      \item store the timing data of every build
      \item use this data to generate the chart
    \end{itemize}
    \pause
    One solution : bash + CSV + gnuplot\\
    \pause
    Another solution : Python + XML + matplotlib
  \end{frame}
  \begin{frame}
    \frametitle{Problem}
    I hadn't used Python before.\\
    \pause
    A good opportunity to learn Python!
  \end{frame}
  \begin{frame}
    \frametitle{Learning Python}
    \begin{itemize}
      \item start with a tutorial
      \item read documentation
      \item ask Google and Stack Overflow
      \item talk to a friend or colleague
    \end{itemize}
  \end{frame}
  \begin{frame}
    \frametitle{Other helpful things}
    \begin{itemize}
      \item check coding style
        \begin{itemize}
          \item commandline : pep8, pylint
          \item online tools : Landscape.io, Scrutinizer
        \end{itemize}
      \item unit testing and test driven development (TDD)
      \item code coverage
      \item automate this with Continuous Integration (CI) : Travis CI
      \item version control : Git, GitHub, ...
    \end{itemize}
  \end{frame}
  \begin{frame}
    \frametitle{Proof of concept}
    Collection of Bash and Python scripts, generating, analysing and visualising timing data:
  \end{frame}

  \begin{frame}
    \frametitle{Lessons learned}
    \begin{itemize}
      \item keep it simple
      \item (re)use as much existing tools as possible
    \end{itemize}
  \end{frame}
  \begin{frame}
   \frametitle{Questions}
    Thanks for your attention!\\
    Questions?
    \vfill
    Contact
    \begin{itemize}
      \item Website : \href{https://buildtimetrend.github.io/}{https://buildtimetrend.github.io/}
      \item Twitter : \href{https://twitter.com/buildtime_trend}{@buildtime\_trend}
    \end{itemize}
  \end{frame}
\end{document}
