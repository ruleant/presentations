\documentclass[14pt]{beamer}
\usepackage{hyperref}
\usepackage{ulem}
\usepackage[T1]{fontenc}
\usetheme{Madrid}

% macro for double dash
\newcommand{\dd}{{\texttt{-{}-}}}

%% Enable license helpers
\graphicspath{{../cc_beamer/}}
\input{../cc_beamer/cc_beamer}

\title[Buildtime Trend]{Buildtime Trend}
\author{Dieter Adriaenssens}
\institute[Buildtime Trend]{@dcadriaenssens - Buildtime Trend}
\date[NewLine 5Apr2015]{NewLine - Ghent\\
April 5th, 2015}
\subject{Buildtime Trend}
\begin{document}
  \begin{frame}
  \titlepage
  \vfill
    \begin{center}
      \CcGroupByNcSa{0.83}{0.95ex}\\[2.5ex]
        {\tiny\CcNote{\CcLongnameByNcSa}}
        \vspace*{-2.5ex}
    \end{center}
  \end{frame}
  \begin{frame}
    \frametitle{Overview}
    \begin{itemize}
      \item Brief introduction to iptables/netfilter
      \item Optimizing configuration with a tutorial use case
      \item Conclusion
    \end{itemize}
  \end{frame}
  \begin{frame}
    \frametitle{What is iptables/netfilter?}
    %(question to audience : who knows/uses iptables?)
    Iptables is a tool for creating the rulesets for netfilter, a packet filtering framework which was introduced in the linux 2.4 kernel
  \end{frame}

  \begin{frame}
    \frametitle{Conclusion}
    \begin{itemize}
      \item use ESTABLISHED state to reduce number of rules
      \item using chains makes your rules easier readable and maintainable
      \item chains can be reused for several rules
      \item chains can be chained together
      \item faster, because it only jumps to a chain when a rule matches
    \end{itemize}
  \end{frame}
  \begin{frame}
   \frametitle{Questions}
    Thanks for your attention!\\
    Questions?
    \vfill
    Contact
    \begin{itemize}
      \item Blog : \href{http://ruleant.blogspot.com/}{http://ruleant.blogspot.com}
      \item Twitter : \href{https://twitter.com/dcadriaenssens}{@dcadriaenssens}
    \end{itemize}
  \end{frame}
\end{document}
