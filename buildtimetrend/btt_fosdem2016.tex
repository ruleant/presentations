\documentclass[14pt]{beamer}
\usepackage{hyperref}
\usepackage{ulem}
\usepackage[T1]{fontenc}
\usetheme{Madrid}

%% Enable license helpers
\graphicspath{ {../cc_beamer/}{pictures/} }
\input{../cc_beamer/cc_beamer}

\logo{
  \includegraphics[scale=.1]{logo_buildtimetrend.png}
}
\title[Buildtime Trend]{Buildtime Trend}
\subtitle{What's trending in your build process}
\author{Dieter Adriaenssens}
\institute[]{Buildtime Trend founder \& developer - @dcadriaenssens}
\date[FOSDEM 2016]{FOSDEM 2016 - Brussels\\
January 31st, 2016}
\subject{Buildtime Trend}
\begin{document}
  \begin{frame}
    \titlepage
    \vfill
    \begin{center}
      \CcGroupByNcSa{0.83}{0.95ex}\\[2.5ex]
        {\tiny\CcNote{\CcLongnameByNcSa}}
        \vspace*{-2.5ex}
    \end{center}
  \end{frame}
  \begin{frame}
    \frametitle{Overview}
    \begin{itemize}
      \item What is Buildtime Trend?
      \item How does it work?
      \item How to start using it
      \item Demo
    \end{itemize}
  \end{frame}
  \begin{frame}
    \frametitle{What is Buildtime Trend?}
    Buildtime Trend is an Open Source application that uses (timing) data to visualise trends in a build process.
  \end{frame}
  \begin{frame}
    \frametitle{It started with an itch}
    I was working on a project that was built on Travis CI:
    \begin{itemize}
      \item some builds took longer than others
      \item no timing information was present in the logs
      \item which stage took longer?
    \end{itemize}
  \end{frame}
  \begin{frame}
    \frametitle{Proof of concept}
    Collection of Bash and Python scripts, generating, analysing and visualising timing data:
    \includegraphics[scale=.45]{example_matplotlib_trend.png}
  \end{frame}
  \begin{frame}
    \frametitle{How it works}
    Service :
    \begin{itemize}
      \item get Travis CI build logfile
      \item parse the (timing) data
      \item store data
      \item visualise using a dashboard with trends
    \end{itemize}
  \end{frame}
  \begin{frame}
    \frametitle{Keen.io}
    \includegraphics[scale=.20]{keenio_workflow.png}\\
    Keen IO's powerful APIs do the heavy lifting for you, so you can gather all the data you want and start getting the answers you need.
  \end{frame}
  \begin{frame}
    \frametitle{First release - dashboard}
    \includegraphics[scale=.45]{example_dashboard.png}
  \end{frame}
  \begin{frame}
    \frametitle{Hosting the service}
    \begin{itemize}
      \item CherryPy turns scripts into webservice
        \begin{itemize}
          \item dashboard
          \item badges
          \item parse buildlog
        \end{itemize}
      \item service hosted on Heroku : \href{https://buildtimetrend.herokuapp.com/}{https://buildtimetrend.herokuapp.com/}
    \end{itemize}
  \end{frame}
  \begin{frame}
    \frametitle{How to use it}
    Trigger the service at the end of a Travis CI build in \textit{.travis.yml} :
    \begin{example}
      \small{notifications:\\
      \ \ webhooks:\\
      \ \ \ \ - https://buildtimetrend.herokuapp.com/travis}
    \end{example}
  \end{frame}
  \begin{frame}
    \frametitle{Business model}
    \begin{itemize}
      \item offering the service costs money
        \begin{itemize}
	  \item hosting the service
          \item storing the data
        \end{itemize}
      \item Github inspired business model : free for Open Source, paying for private repos
      \item \href{https://keen.io}{Keen.io} offered to host data for free for Open Source projects. \textbf{Thanks, guys!}
      \item Further development with real data gathering : better feedback
    \end{itemize}
  \end{frame}
  \begin{frame}
    \frametitle{Future development}
    \begin{itemize}
      \item caching for badges and dashboard charts
      \item support private Github repos
      \item support other CI platforms (Jenkins, ...)
      \item more and improved metrics and trends
    \end{itemize}
  \end{frame}
  \begin{frame}
    \frametitle{Contributions welcome}
    \begin{itemize}
      \item use and test the service
      \item report bugs
      \item suggest improvements
      \item clone the project and send pull requests
      \begin{itemize}
        \item implement a feature
        \item fix a bug
        \item add a new chart
        \item improve dashboard layout
        \item \ldots
      \end{itemize}
    \end{itemize}
  \end{frame}
  \begin{frame}
    \frametitle{Demo}
    \begin{itemize}
      \item Service : \href{https://buildtimetrend.herokuapp.com/}{\small{https://buildtimetrend.herokuapp.com/}}
      \item Dashboard : \href{https://buildtimetrend.herokuapp.com/dashboard/buildtimetrend/python-lib/index.html}{\small{https://buildtimetrend.herokuapp.com/dashboard/buildtimetrend/python-lib}}
      \item Badges : \href{https://github.com/buildtimetrend/service\#badge-examples}{\small{https://github.com/buildtimetrend/service\#badge-examples}}
    \end{itemize}
  \end{frame}
  \begin{frame}
    \frametitle{Acknowledgements}
    Big thanks to
    \begin{itemize}
      \item The nice people of Keen.io, for their invaluable support!
      \item the Open Source projects testdriving the service
      \item all the \href{https://github.com/buildtimetrend/python-lib/wiki/Credits}{services} that power the project
    \end{itemize}
  \end{frame}
  \begin{frame}
   \frametitle{Questions}
    Thanks for your attention!\\
    Questions?
    \vfill
    Dieter Adriaenssens - \href{https://twitter.com/dcadriaenssens}{\small{@dcadriaenssens}}
    \vfill
    Presentation available on \href{https://ruleant.github.io/presentations/}{\small{https://ruleant.github.io/presentations/}}
    \vfill
    Buildtime Trend
    \begin{itemize}
      \item Website : \href{https://buildtimetrend.github.io/}{\small{https://buildtimetrend.github.io/}}
      \item Service : \href{https://buildtimetrend.herokuapp.com/}{\small{https://buildtimetrend.herokuapp.com/}}
      \item Twitter : \href{https://twitter.com/buildtime_trend}{\small{@buildtime\_trend}}
    \end{itemize}
  \end{frame}
\end{document}
